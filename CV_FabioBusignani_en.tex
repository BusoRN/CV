\documentclass[11pt,a4paper]{moderncv}

% moderncv themes
%\moderncvtheme[blue]{casual}                 % optional argument are 'blue' (default), 'orange', 'red', 'green', 'grey' and 'roman' (for roman fonts, instead of sans serif fonts)
\moderncvtheme[green]{classic}                % idem

\usepackage[T1]{fontenc}
\usepackage{multicol}
% character encoding
\usepackage[utf8x]{inputenc}                   % replace by the encoding you are using
\usepackage[italian]{babel}

% adjust the page margins
\usepackage[scale=0.8]{geometry}
\recomputelengths                             % required when changes are made to page layout lengths

\fancyfoot{} % clear all footer fields
\fancyfoot[LE,RO]{\thepage}           % page number in "outer" position of footer line
\fancyfoot[RE,LO]{\footnotesize } % other info in "inner" position of footer line

% personal data
\firstname{Fabio}
\familyname{\\ \\Busignani}
\title{Curriculum Vit\ae}               % optional, remove the line if not wanted
\address{via Longiano 26, Santa Giustina}{47922 Rimini, Italy}    % optional, remove the line if not wanted
\mobile{+39 333 1963542}                    % optional, remove the line if not wanted
\phone{+39 0541 681253}                      % optional, remove the line if not wanted
%\fax{312 996 1491}                          % optional, remove the line if not wanted
\email{fbusignani@linux.com}                      % optional, remove the line if not wanted
%\extrainfo{additional information (optional)} % optional, remove the line if not wanted
\photo[84pt]{fototessera}          % '64pt' is the height the picture must be resized to and 'picture' is the name of the picture file; optional, remove the line if not wanted
\extrainfo{Skype : \textit{fabio.busignani}}
\extrainfo{\href{https://it.linkedin.com/in/FabioBusignani}{LinkedIn} }
%\quote{"Winter is coming" -- House of Stark}                 % optional, remove the line if not wanted

%----------------------------------------------------------------------------------
%            content
%----------------------------------------------------------------------------------
\begin{document}
\maketitle
\small 
\vspace{-5mm}

%Section
%\section{Info}
%\cvcomputer{Born}{\small September 8th, 1990 in Rimini (Italy)\normalsize}{Citizenship}{\small Italian\normalsize}
%\cvcomputer{Driver's License}{\small B, A\normalsize}{}{\small \url{}\normalsize}


%Section
%\section{Desidered Employment and Current Skills}
%\cvline{}{\small I have always loved the electronic's world, long before I started my studies in electronics.
%My interest for electronics goes beyond the academic studies: in fact, I spend a good part of my free time experimenting and prototyping with microcontrollers and embedded systems boards. Everything is confirmed by my excellent university results. In the last two academic years I'm specializing in electronic systemse.}
%\section{About Me}
%\cvline{}{Fabio is a new graduate in Electronic Engineering from the Polytechnic University of Turin.  While he achieved high marks in school, his interest for electronics goes
%	well beyond the academic setting. Indeed in his free time, he loves to experiment and prototype
%	with microcontrollers and embedded systems boards as well as deepen some interesting topic. \newline
%	After he took a bachelor degree in Electronics, Informatics and Telecommunications Engineering from the University of Bologna he choose to moved to Turin enrolling at the formal specialization in Electronic Systems, because it allowed him to have a sound electronic knowledge, focusing on digital electronics where he made some experience in every level of the design workflow of digital hardware implementation and SoC design. To carry out his Master Thesis, Fabio joined Khademhosseini Lab, a research group of Harvard-MIT Health Science and Technology. During this 6-month experience  he became
%	immediately involved in a project on customizing Google Glass for its application in
%	biomedicine. To fulfill this aim he designed from the scratch a completely new system 
%	based on the BeagleBone Black. This highly integrated platform can seamlessly collect data from all the biosensors including the pH, temperature, and the microscope, which are then directly
%	transmitted to the Google App Engine and finally downloaded and showed on Google Glass. More interestingly,
%	Fabio also added the capability to control the experiments using the Google Glass.  In another project, he designed a miniature function generator  used to stimulate the cardiomyocytes cultured on three-dimensional scaffolds to enhance the alignment and maturation of the cells for engineering better functional cardiac tissues. \newline
%	Now Fabio is working on Sherlock project, where he collaborates with the team as hardware and firmware designer. }

%\section{Skills}
%\cvlistdoubleitem{English (Professional Proficiency)}{PCB Design}
%\cvlistdoubleitem{ Microcontrollers}{Embedded Systems}
%\cvlistdoubleitem{Low-Power Design Techniques}{Android Programming}

\section{Desidered Employment and Current Skills}
\cvline{}{\textbf{Hardware Engineer - ASIC Designer}\newline
I have always loved the world of electronics, long before I started my studies in electronics. While I achieved high marks in school, my interest for electronics goes well beyond the academic setting. Indeed in my free time, I love to experiment with microcontrollers and embedded systems boards as well as deepen some interesting topics. \newline
My course of study allowed me to have a strong electronic knowledge, focusing on digital electronics where I made some experience in every level of the design workflow of digital hardware implementation and SoC design. %Indeed my dream is to become a digital designer. 
%What I am looking for is a working environment where I can grow both professionally and personally allowing me to follow my dream.
}

\section{Technical skills} 
\cvline{Languages}{\small VHDL, C/C++, SystemC,   JAVA, C\#,  UML, MIPS Assembly, PIC Assembly, Python, \LaTeX}
\cvline{Operating Systems}{Linux, Android, Windows} 
\cvline{Software Tools}{Modelsim, Synopsys Design Compiler, Synopsys Design Vision, Quartus II, Cadence Design Environment, Multisim \& Ultiboard, NI
	LabVIEW, Matlab + Simulink, Android Studio, Qt, mbed.org, IAR, MPLAB, Code Composer
	Studio}
\cvline{Hardware Platforms}{Electronic bench equipment, microPIC, MSP430, FRDM-KL25Z, Beaglebone Black, RaspberryPI, Altera Cyclon II, STM32F401}

%Section
\section{Education}
\cventry{Oct. 2012 - Mar. 2015}{Master Degree in Electronic Engineering}{\href{http://www.polito.it/}{Polytechnic University of Turin}}{Turin}{Italy}{Specialization in Electronic Systems\\
	\textit{Thesis Title}: Google Glass Assisted Data Visualization and Monitoring for Organs-on-a-Chip and Biomedical Applications.\\
	\textit{Grade}: 110/110
	%\textit{Learning Activities:}\\
	%\vspace{-3mm}
}
%\begin{cvcolumns}
%
%\cvcolumn[0.14]{\small Code}{ \footnotesize 01NNIOQ \\ 01NNLOQ \\ 01NOHOQ \\ 01NVHOQ \\ 01NVIOQ \\ 01NVLOQ \\ 02NQWOQ \\ 01NOKOQ \\ 02GQCOQ  \\ 01OBCOQ \\ 01OUZOQ \\ 01NNNOQ \\ 04JEZOQ}
%\cvcolumn[0.38]{\small Name of exam}{\footnotesize Digital Integrated Systems\\ High Speed Electron Devices\\ Low Power Electronic Systems\\ Guiding Electromagnetic Systems\\ Measurement Systems and Sensors\\Digital Microelectronics\\ Numerical Methods\\ Electronic Systems Engineering \\ Integrated Systems Architecture  \\ Codesign Methods and Tools \\ Model-based Software Design \\ Analog and Power Electronics \\ Operating Systems }
%\cvcolumn{\small Grade}{ \footnotesize 26/30 \\ 30/30 \\ 30L/30 \\ 30L/30 \\ 27/30 \\ 30/30 \\ 25/30\\ 30/30 \\ 29/30\\ 30/30 \\ 30L/30 \\ 24/30 \\ 30/30}
%\cvcolumn{\small Passed on}{\footnotesize 30/01/2013 \\ 22/02/2013 \\ 25/07/2013 \\ 03/07/2013 \\ 27/02/2013 \\ 28/06/2013 \\ 19/02/2013 \\ 31/01/2014 \\ 30/01/2014 \\ 20/06/2014 \\ 18/07/2014 \\ 26/02/2014 \\ 14/07/2013}
%\cvcolumn{\small Credits}{\footnotesize 10.00\\6.00\\6.00\\8.00\\8.00\\6.00\\6.00\\6.00\\6.00 \\6.00\\6.00\\10.00\\6.00}
%\end{cvcolumns}

%\begin{multicols}{1}





%\begin{itemize}
%\item Digital Integrated Systems;
%\item Numerical Methods;
%\item High Speed Electron Devices;
%\item Measurment Systems and Sensors;
%\item Digital Microelectronics;
%\item Guiding Electromagnetic Systems;
%\item Low Power Electronic Systems.
%\end{itemize}
%\end{multicols}

\cventry{Sep. 2009 - Oct. 2012}{Bachelor Degree in Electronics, Informatics and Telecommunications Engineering}{\href{http://http://www.ingegneriarchitettura.unibo.it/}{Seconda facoltà di ingegneria con sede a Cesena - University of Bologna}}{Cesena}{Italy}{\textit{Thesis title}: Design of a microcontroller system which controls an electrodynamic shaker.
	%\textit{Learning Activities Successfully Completed:}
	%\vspace{-3mm}
	%\begin{multicols}{2}
	%\begin{itemize}
	%\item Advanced Topics in Electronics;
	%\item Automatic Controls;
	%\item Basics Principles of Chemistry;
	%\item Digital Design Principles and Computer Architecture;
	%\item Economics and Business Organisation;
	%\item Electronic Measurements;
	%\item Electronics;
	%\item Electronics of Digital Systems;
	%\item Electrotechnics;
	%\item Foreign Language: English;
	%\item Foundations of Informatics A;
	%\item General Physics A;
	%\item General Physics B;
	%\item Geometry and Algebra;
	%\item Industrial Electronics;
	%\item Internship;
	%\item Laboratory of Telecommunications;
	%\item Mathematical Analysis A;
	%\item Mathematical Analysis B;
	%\item Operations Research;
	%\item Propagation and Electromagnetic Fields (Integrated Course);
	%\item Signal Processing;
	%\item Telecommunication techniques and applications;
	%\item Telecommunications.
	%\end{itemize}
	%\end{multicols}
} 

\cventry{Sep. 2004 - July 2009}{Electronic and Telecommunications Technician}{\href{http://www.itisrn.it/}{I.T.I.S. Leonardo da Vinci}}{Rimini}{Italy}{Italian secondary school diploma}

%\vspace{-3mm}
%Section

\section{Academic Projects}
%%Qui di seguito verranno elencati i progetti da me svolti durante gli studi per il corretto conseguimento di alcuni corsi.  \\
\cventry{June 2014}{Multithreaded Blowfish Algorithm Implementation}{}{}{}{
	\href{https://github.com/BusoRN/blowfish-OS}{Github link.}}

\cventry{January 2014}{Design of a 4-state ACS (Add-Compare-Select) with SystemC}{}{}{}{Design covered the following steps:
	\begin{itemize}
		\item ACS design;
		\item SystemC implementation;
		\item Matlab implementation used to verify the results;
		\item First simulation with G++ compiler and GTKwave viewer;
		\item Second simulation with SystemC Modelsim;
		\item Third simulation with  Modelsim mixed SystemC-Verilog  using a given implementation of the ACS.
	\end{itemize} \href{https://github.com/BusoRN/ISA-projects/tree/master/Lab3}{Github link.}}

\cventry{December 2013}{ASIP design based on Transport Triggered Architecture using TCE}{}{}{}{Design of an architecture which implements DCT, using \textit{TTA} architecture.\newline \href{https://github.com/BusoRN/ISA-projects/tree/master/Lab2}{Github link.}}

\cventry{November 2013}{FIR ASIC Implementation}{}{}{}{Design covered the following steps:
	\begin{itemize}
		\item Behavior design with Matlab;
		\item VHDL implementation;
		\item Testbench implementation and simulation with Modelsim;
		\item Switching activity-based power consumption estimation;
		\item ASIC Place\&Route;
		\item Post Place\&Route simulation and switching activity-based power consumption estimation.
	\end{itemize} \href{https://github.com/BusoRN/ISA-projects/tree/master/Lab1}{Github link.}}


\cventry{July 2013}{
	Low-power serial interface for SoC functional units communication}{}{}{}{
	The aim was to make a very low power interface, and in order to do this,  certain low power design methods were exploited, such as clock gating and state encoding. Synthesis was made with \textit{Synopsys Design Compiler}. \newline \href{https://github.com/BusoRN/Low-Power-Project}{Github link.}}


\cventry{Jenuary 2013}{Butterfly Design using Microcode Technique}{}{}{}{Design of a processing element which implements a \textit{Butterfly}. In this project several hardware constraints were given, and they were solved through the adoption of the \textit{folding} technique. }

\cventry{December 2012}{Wishbone Implementation}{}{}{}{Design covered the following steps:
	\begin{itemize}
		\item VHDL implementation;
		\item Testbench implementation and simulation with Modelsim;
		\item Synthesis using Quartus;
		\item Hardware test using Altera Cyclone II.
	\end{itemize} }




%\cventry{2012-2013}{Digital Integrated Systems}{Polytechnic University of Turin}{Torino}{Italy}{Design of an interface which manages the communications in a master-multi slave system. For this project a typical standard protocol of System-on-Chip communications, \textit{Whisbone},was exploited. The EDA softwares used are Modelsim and Quarts.
%\newline 
%Design of a processing element which implements a \textit{Butterfly}. In this project several hardware constraints were given, and they were solved through the adoption of the \textit{folding} technique.  }
%\cventry{2012-2013}{Low Power Electronic Systems}{Polytechnic University of Turin}{Torino}{Italy}{ %Laboratory experiences:
%	\begin{enumerate}
%		\item 	\textit{Power Estimation: probabilistic techniques}, used Modelsim to perform a power analysis of some combinatorial circuit described in VHDL.
%		\item  \textit{FSM State Assignment and VHDL Synthesis} using Synopsys.
%		\item  \textit{Clock gating, pipelining and parallelizing}, using Synopsys to synthesis and Modelsim for simulation. 
%		\item \textit{Bus encoding}, performed power analysis of a data transmission with different encoding methods: bus invert, transition-based, gray, T0.
%		\item \textit{Leakage}: using spice for characterizing cells
%		and pen\&paper for memory organization.
%	\end{enumerate}
%	Final Project:\newline
%	Design of a serial interface that allows the interaction between a control unit and a computing unit. The interface must be specifically embedded inside a system which controls the supply noise inside integrated circuits. The aim was to make a very low power interface, and in order to do this,  certain low power design methods were exploited, such as clock gating and state encoding. Synthesis was made with \textit{Synopsys Design Compiler}. \newline \href{https://github.com/BusoRN/Low-Power-Project}{Github link.}}
%\cventry{2012-2013}{Measurement Systems and Sensors}{Polytechnic University of Turin}{Turin}{Italy}{Development of severals measuring benches exploiting \textit{NI DAQ Board}. This Data acquisition board was driven through forms written in C\#.}
%\cventry{2012-2013}{Digital Microelectronic}{Polytechnic University of Turin}{Turin}{Italy}{Laboratory practices using \textit{Cadence Allegro} to design some masks of digital gates.}
%
%\cventry{June 2013}{Stepped impedance RF low-pass filter}{}{}{}{Designed a stepped impedance RF low-pass filter using \textit{AWR microwave office}. \newline \href{https://github.com/BusoRN/Guiding-project}{Github link.}}
%

\section{Working Experience}

\cventry{Apr. 2015 - Present}{Firmware Designer}{\href{http:http://www.sherlock.bike/}{Sherlock Bike}}{Turin}{Italy}{Sherlock is a GPS-based anti-theft device connected to a smartphone app that allows cyclists to precisely locate their bicycles and retrieve them in case of theft.}
\cventry{Aug. 2014 - Feb. 2015}{Research Trainee}{\href{http://www.tissueeng.net/lab/}{Khademhosseini Lab (Harvard-MIT Health Sciences and Technology )}}{Cambridge, MA}{United States of America}{Designed and developed a custom user interface on Google Glass for simultaneous recording of
	biosensing data (T, pH) and microscopy images/videos as well as remote control of microfluidic
	valves for organ-on-a-chip applications.
	In order to fulfill this aim I have designed and created a complete system
	which is based on Beaglebone Black.}

\cventry{July 2012 - Sep. 2012}{Internship}{\href{http://www.4ts.it/}{Fortech s.r.l.}}{Rimini}{Italy}{Worked with the company's electronic engineers, learning about the hardwares present in payment systems of fuel service stations. }
\cventry{2007 - 2010}{Summer Jobs}{\href{}{TES s.r.l. and FM s.n.c.}}{Rimini}{Italy}{Summer job as apprentice electrician and photovoltaic technician. }


%Section
%\section{Languages}
%\hspace{25mm}\small  Self-assessment European level \href{http://europass.cedefop.europa.eu/en/resources/european-language-levels-cefr}{CEFR} (C2 maximum evaluation)\normalsize
%\vspace{5mm}
%
%\begin{tabular}{p{67mm} p{40mm} p{40mm} p{20mm}}
%& \textbf{Comprehension} & \textbf{Speaking} & \textbf{Writing} \\
%\end{tabular}
%
%\begin{tabular}{p{67mm} p{20mm} p{20mm} p{20mm} p{20mm} p{20mm}}
%& Listening & Reading & Interaction & Production & \\
%\end{tabular}
%
%\vspace{3mm}
%\cvlanguage{Italian}{Mother Tongue}{
%	\begin{tabular}{ p{20mm} p{20mm} p{20mm} p{20mm} p{21mm}}
%		C2 & C2 & C2 & C2 & C2
%	\end{tabular}}
%\cvlanguage{English}{Good Knowledge}{
%	\begin{tabular}{ p{20mm} p{20mm} p{20mm} p{20mm} p{21mm}}
%		C1 & C1 & B2 & B2 & B2
%	\end{tabular}
%}
%%Section
%
%

\vspace{-3mm}
%\section{Social and Organization Skills}
%\cvline{}{During my 6-month experience in the United States I greatly improved my ability to communicate and
%	to work in a multicultural working environment with colleagues with different backgrounds (chemicals, bioengineers,  biologists). As a meticulous person, I work hard and pay much attention to details, always
%	aiming to improve myself. I learn new skills quickly and am eager
%	to learn from others. I am able to work by myself as well as part of a team. \newline
%	I am an active guy, indeed in my free time I really love play sports and train, I also like spend time with friends and play board game.  }

\vspace{-3mm}
%Section
%\section{Interests and Hobbies}
%\cvline{}{\small I really love practicing sports during my free time, especially mountain ones such as skiing and trekking. I enjoy watching telefilms and listening to music (mostly rock). I like reading technical books and magazines in order to play with evaluation and development boards like RasperryPI, TI LauchPad and Freescale Freedom Platform.}

\vspace{-3mm}



\end{document}